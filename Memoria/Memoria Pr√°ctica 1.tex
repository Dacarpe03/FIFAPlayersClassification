%\documentclass[a4paper,11pt]{article}
\documentclass{article}
%%%%%%%%%%%%%%%%%%%%%%%%%%%%%%%%%%%%%%%%%%%%%%%%%%%%%%%%%%%%%%%%%%%%%%%%%%%%%%%%%%%%%%%%%%%%%%%%%%%%%%%%%%%%%%%%%%%%%%%%%%%%
\usepackage{graphics,graphicx}
\usepackage{amsmath,amssymb,graphics,graphicx}
\usepackage[ansinew]{inputenc}
\usepackage[usenames,dvipsnames]{color}

\graphicspath{ {Imagenes/} }

\usepackage{natbib}

\bibpunct{(}{)}{;}{a}{,}{,}

\textheight 24cm \textwidth 17cm \topmargin-2cm
%% \evensidemargin   -0.25cm
\oddsidemargin-0.2cm
%\pagestyle{empty}
\renewcommand{\baselinestretch}{1}

\begin{document}

\title{Pr\'actica 1: FIFA Players Classification}

\author{{Daniel Carmona Pedrajas}}

\date{}
\maketitle

%\title{}

%\address{}

\section{Arquitectura 1: Shallow Neural Network}
	Comenzamos los experimentos usando la siguiente arquitectura:
	\begin{enumerate}
		\item Capa densa con 512 neuronas.
	\end{enumerate}
	\subsection{Experimento 1}
		Usamos la siguiente configuraci\'on:
		\begin{table}[h]
			\begin{center}
				\begin{tabular}{ c | c | c | c | c | c | c}
					\textbf{Epochs} & \textbf{Learning rate} & \textbf{Batch size} & \textbf{Activation} & \textbf{Loss} & \textbf{Optimizer} & \textbf{Regularization} \\ \hline
					100 & 0.1 & 512 & ReLu & Categorical Crossentropy & SGD & None
				\end{tabular}
				\caption{Hyperparameters for Experiment 1 with Architecture 1}
				\label{tab:exp1-1}
			\end{center}
		\end{table}
\end{document}
